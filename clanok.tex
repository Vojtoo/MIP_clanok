% Metódy inžinierskej práce

\documentclass[10pt,twoside,slovak,a4paper]{article}

\usepackage[slovak]{babel}
%\usepackage[T1]{fontenc}
\usepackage[IL2]{fontenc} % lepšia sadzba písmena Ľ než v T1
\usepackage[utf8]{inputenc}
\usepackage{graphicx}
\usepackage{url} % príkaz \url na formátovanie URL
\usepackage{hyperref} % odkazy v texte budú aktívne (pri niektorých triedach dokumentov spôsobuje posun textu)
\usepackage{graphicx}
\graphicspath{ {D:\Vskola\MIP} }
\usepackage{cite}
%\usepackage{times}

\pagestyle{headings}

\title{Hry o prežitie\thanks{Semestrálny projekt v predmete Metódy inžinierskej práce, ak. rok 2022/2023, vedenie: Vladimír Mlynarovič}} % meno a priezvisko vyučujúceho na cvičeniach

\author{Vojtech Babinský\\[2pt]
	{\small Slovenská technická univerzita v Bratislave}\\
	{\small Fakulta informatiky a informačných technológií}\\
	{\small \texttt{xbabinskyv@stuba.sk}}
	}

\date{\small 6. november 2022} % upravte



\begin{document}

\maketitle

\begin{abstract}

Vďaka pokroku, sa už väčšina populácie v rozvinutých krajinách nemusí zaoberať poľnohospodárstvom, lovom, zberom a podobnou ťažkou fyzickou prácou. Tieto aktivity, dnes vykonáva, vzhľadom na ostatok obyvateľstva, len hŕstka ľudí. 

Niektorí ľudia, však vo svojom voľnom čase siahajú po hrách simulujúcich tieto činnosti, často v extrémnych podmienkach, kde každá chyba môže mať nemilý vplyv na ďalšie hranie. Ide teda o špecifický prípad, kde na rozdiel od väčšiny iných hier, hráč nie je nezraniteľným hrdinom, ale súčasťou prostredia, v ktorom zápasí s hladom, zvieratami/príšerami, počasím, inými hráčmi atď. 

Cieľom článku je zoznámiť čitateľa s tzv. hrami o prežite (ang. Survival games).Pokúsi sa opísať kritéria, ktoré musia hry spĺňať, aby mohli byť zaradené do tohto žánru, poskytne príklady  úspešných herných predstaviteľov tejto kategórie a predstaví aj jej rôzne subžánre. 
\end{abstract}



\section{Úvod}



\section{Moja nová časť}



\section{Nejaká časť} \label{nejaka}

\cite{Reid}







\section{Záver} \label{zaver} % prípadne iný variant názvu



%\acknowledgement{Ak niekomu chcete poďakovať\ldots}


% týmto sa generuje zoznam literatúry z obsahu súboru literatura.bib podľa toho, na čo sa v článku odkazujete
\bibliography{literatura}
\bibliographystyle{plain} % prípadne alpha, abbrv alebo hociktorý iný

\end{document}
